\chapter{Conclusão}
% ---
\section{Diferentes idiomas e hifenizações}
\label{sec-hifenizacao}
% ---

Para usar hifenizações de diferentes idiomas, inclua nas opções do documento o
nome dos idiomas que o seu texto contém. Por exemplo (para melhor
visualização, as opções foram quebras em diferentes linhas):

\begin{verbatim}
\documentclass[
	12pt,
	openright,
	twoside,
	a4paper,
	english,
	french,
	spanish,
	brazil
	]{abntex2}
\end{verbatim}

O idioma português-brasileiro (\texttt{brazil}) é incluído automaticamente pela
classe \textsf{abntex2}. Porém, mesmo assim a opção \texttt{brazil} deve ser
informada como a última opção da classe para que todos os pacotes reconheçam o
idioma. Vale ressaltar que a última opção de idioma é a utilizada por padrão no
documento. Desse modo, caso deseje escrever um texto em inglês que tenha
citações em português e em francês, você deveria usar o preâmbulo como abaixo:

\begin{verbatim}
\documentclass[
	12pt,
	openright,
	twoside,
	a4paper,
	french,
	brazil,
	english
	]{abntex2}
\end{verbatim}

A lista completa de idiomas suportados, bem como outras opções de hifenização,
estão disponíveis em \citeonline[p.~5-6]{babel}.

Exemplo de hifenização em inglês\footnote{Extraído de:
\url{http://en.wikibooks.org/wiki/LaTeX/Internationalization}}:

\begin{otherlanguage*}{english}
\textit{Text in English language. This environment switches all language-related
definitions, like the language specific names for figures, tables etc. to the other
language. The starred version of this environment typesets the main text
according to the rules of the other language, but keeps the language specific
string for ancillary things like figures, in the main language of the document.
The environment hyphenrules switches only the hyphenation patterns used; it can
also be used to disallow hyphenation by using the language name
`nohyphenation'.}
\end{otherlanguage*}

Exemplo de hifenização em francês\footnote{Extraído de:
\url{http://bigbrowser.blog.lemonde.fr/2013/02/17/tu-ne-tweeteras-point-le-vatican-interdit-aux-cardinaux-de-tweeter-pendant-le-conclave/}}:

\begin{otherlanguage*}{french}
\textit{Texte en français. Pas question que Twitter ne vienne faire une
concurrence déloyale à la traditionnelle fumée blanche qui marque l'élection
d'un nouveau pape. Pour éviter toute fuite précoce, le Vatican a donc pris un
peu d'avance, et a déjà interdit aux cardinaux qui prendront part au vote
d'utiliser le réseau social, selon Catholic News Service. Une mesure valable
surtout pour les neuf cardinaux – sur les 117 du conclave – pratiquants très
actifs de Twitter, qui auront interdiction pendant toute la période de se
connecter à leur compte.}
\end{otherlanguage*}

Pequeno texto em espanhol\footnote{Extraído de:
\url{http://internacional.elpais.com/internacional/2013/02/17/actualidad/1361102009_913423.html}}:

\foreignlanguage{spanish}{\textit{Decenas de miles de personas ovacionan al pontífice en su
penúltimo ángelus dominical, el primero desde que anunciase su renuncia. El Papa se
centra en la crítica al materialismo}}.

O idioma geral do texto por ser alterado como no exemplo seguinte:

\begin{verbatim}
  \selectlanguage{english}
\end{verbatim}

Isso altera automaticamente a hifenização e todos os nomes constantes de
referências do documento para o idioma inglês. Consulte o manual da classe
\cite{abntex2classe} para obter orientações adicionais sobre internacionalização de
documentos produzidos com \abnTeX.

A \autoref{sec-citacao} descreve o ambiente \texttt{citacao} que pode receber
como parâmetro um idioma a ser usado na citação.

% ---
\section{Consulte o manual da classe \textsf{abntex2}}
% ---

Consulte o manual da classe \textsf{abntex2} \cite{abntex2classe} para uma
referência completa das macros e ambientes disponíveis. 

Além disso, o manual possui informações adicionais sobre as normas ABNT
observadas pelo \abnTeX\ e considerações sobre eventuais requisitos específicos
não atendidos, como o caso da \citeonline[seção 5.2.2]{NBR14724:2011}, que
especifica o espaçamento entre os capítulos e o início do texto, regra
propositalmente não atendida pelo presente modelo.

% ---
\section{Referências bibliográficas}
% ---

A formatação das referências bibliográficas conforme as regras da ABNT são um
dos principais objetivos do \abnTeX. Consulte os manuais
\citeonline{abntex2cite} e \citeonline{abntex2cite-alf} para obter informações
sobre como utilizar as referências bibliográficas.

%-
\subsection{Acentuação de referências bibliográficas}
%-

Normalmente não há problemas em usar caracteres acentuados em arquivos
bibliográficos (\texttt{*.bib}). Porém, como as regras da ABNT fazem uso quase
abusivo da conversão para letras maiúsculas, é preciso observar o modo como se
escreve os nomes dos autores. Na ~\autoref{tabela-acentos} você encontra alguns
exemplos das conversões mais importantes. Preste atenção especial para `ç' e `í'
que devem estar envoltos em chaves. A regra geral é sempre usar a acentuação
neste modo quando houver conversão para letras maiúsculas.

\begin{table}[htbp]
\caption{Tabela de conversão de acentuação.}
\label{tabela-acentos}

\begin{center}
\begin{tabular}{ll}\hline\hline
acento & \textsf{bibtex}\\
à á ã & \verb+\`a+ \verb+\'a+ \verb+\~a+\\
í & \verb+{\'\i}+\\
ç & \verb+{\c c}+\\
\hline\hline
\end{tabular}
\end{center}
\end{table}
