\chapter{Processamento dos Dados}
% ---
\section{Enumerações: alíneas e subalíneas}
% ---

\index{alíneas}\index{subalíneas}\index{incisos}Quando for necessário enumerar
os diversos assuntos de uma seção que não possua título, esta deve ser
subdividida em alíneas \cite[4.2]{NBR6024:2012}:

\begin{alineas}

  \item os diversos assuntos que não possuam título próprio, dentro de uma mesma
  seção, devem ser subdivididos em alíneas; 
  
  \item o texto que antecede as alíneas termina em dois pontos;
  \item as alíneas devem ser indicadas alfabeticamente, em letra minúscula,
  seguida de parêntese. Utilizam-se letras dobradas, quando esgotadas as
  letras do alfabeto;

  \item as letras indicativas das alíneas devem apresentar recuo em relação à
  margem esquerda;

  \item o texto da alínea deve começar por letra minúscula e terminar em
  ponto-e-vírgula, exceto a última alínea que termina em ponto final;

  \item o texto da alínea deve terminar em dois pontos, se houver subalínea;

  \item a segunda e as seguintes linhas do texto da alínea começa sob a
  primeira letra do texto da própria alínea;
  
  \item subalíneas \cite[4.3]{NBR6024:2012} devem ser conforme as alíneas a
  seguir:

  \begin{alineas}
     \item as subalíneas devem começar por travessão seguido de espaço;

     \item as subalíneas devem apresentar recuo em relação à alínea;

     \item o texto da subalínea deve começar por letra minúscula e terminar em
     ponto-e-vírgula. A última subalínea deve terminar em ponto final, se não
     houver alínea subsequente;

     \item a segunda e as seguintes linhas do texto da subalínea começam sob a
     primeira letra do texto da própria subalínea.
  \end{alineas}
  
  \item no \abnTeX\ estão disponíveis os ambientes \texttt{incisos} e
  \texttt{subalineas}, que em suma são o mesmo que se criar outro nível de
  \texttt{alineas}, como nos exemplos à seguir:
  
  \begin{incisos}
    \item \textit{Um novo inciso em itálico};
  \end{incisos}
  
  \item Alínea em \textbf{negrito}:
  
  \begin{subalineas}
    \item \textit{Uma subalínea em itálico};
    \item \underline{\textit{Uma subalínea em itálico e sublinhado}}; 
  \end{subalineas}
  
  \item Última alínea com \emph{ênfase}.
  
\end{alineas}

% ---
\section{Espaçamento entre parágrafos e linhas}
% ---

\index{espaçamento!dos parágrafos}O tamanho do parágrafo, espaço entre a margem
e o início da frase do parágrafo, é definido por:

\begin{verbatim}
   \setlength{\parindent}{1.3cm}
\end{verbatim}

\index{espaçamento!do primeiro parágrafo}Por padrão, não há espaçamento no
primeiro parágrafo de cada início de divisão do documento
(\autoref{sec-divisoes}). Porém, você pode definir que o primeiro parágrafo
também seja indentado, como é o caso deste documento. Para isso, apenas inclua o
pacote \textsf{indentfirst} no preâmbulo do documento:

\begin{verbatim}
   \usepackage{indentfirst}      % Indenta o primeiro parágrafo de cada seção.
\end{verbatim}

\index{espaçamento!entre os parágrafos}O espaçamento entre um parágrafo e outro
pode ser controlado por meio do comando:

\begin{verbatim}
  \setlength{\parskip}{0.2cm}  % tente também \onelineskip
\end{verbatim}

\index{espaçamento!entre as linhas}O controle do espaçamento entre linhas é
definido por:

\begin{verbatim}
  \OnehalfSpacing       % espaçamento um e meio (padrão); 
  \DoubleSpacing        % espaçamento duplo
  \SingleSpacing        % espaçamento simples	
\end{verbatim}

Para isso, também estão disponíveis os ambientes:

\begin{verbatim}
  \begin{SingleSpace} ...\end{SingleSpace}
  \begin{Spacing}{hfactori} ... \end{Spacing}
  \begin{OnehalfSpace} ... \end{OnehalfSpace}
  \begin{OnehalfSpace*} ... \end{OnehalfSpace*}
  \begin{DoubleSpace} ... \end{DoubleSpace}
  \begin{DoubleSpace*} ... \end{DoubleSpace*} 
\end{verbatim}

Para mais informações, consulte \citeonline[p. 47-52 e 135]{memoir}.
